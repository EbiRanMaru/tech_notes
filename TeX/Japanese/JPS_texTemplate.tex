%%%%%%%%% JPS abstruct %%%%%%%%%%%%%%%%%%%%%%%%%%%%%%%%%%%%%%%%%%
\documentclass[12pt,a4paper]{jsarticle}

%%%%%%%%% packages %%%%%%%%%%%%%%%%%%%%%%%%%%%%%%%%%%%%%%%%%%%%%%
\usepackage[dvipdfmx]{graphicx} % Include figure files
\usepackage[%                           % 余白の設定
%mag=1400,%                              jarticle の場合(14ptに)
dvipdfm,truedimen,%
top=30truemm,bottom=20truemm,%
left=20truemm,right=20truemm]{geometry}
\pagestyle{empty}

%%%%%%%%% header %%%%%%%%%%%%%%%%%%%%%%%%%%%%%%%%%%%%%%%%%%%%%%%%
\begin{document}
\vspace{-5pt}
\begin{center}
{\gt \Large 講演概要集原稿の書き方 }\\[14pt]

{\gt \large 日本科技大$^A$ 帝都大理工$^B$ \\ 止田次郎$^A$, 潟川 学$^B$}\\[5pt]

{\large \bf How to Write an Abstract for the JPS Meeting}\\[5pt]

{\large \it $^A$Dept. of Phys., Teito Univ. $^B$Dept. of Phys, Nihon Univ. of Tech.}\\

{\large \bf J. Tomeda$^A$ and M. Gatagawa$^B$}
\end{center}

\vspace{10pt}
%%%%%%%%% main %%%%%%%%%%%%%%%%%%%%%%%%%%%%%%%%%%%%%%%%%%%%%%%%%%

この講演概要集原稿はtexのtemplateを用いて用意された。\\
以下に挙げる決まりを守って頂ければ、最終判断は著者にあるので、多少の違いはあっても結構である。

\begin{itemize}
\item 大会名・講演番号・ページ数を入れるため上下にある程度空白を入れる。
\item タイトル、所属、氏名、英語表記は上記のような配置で記載
\item 講演番号は記載しない。
\item 枠はつけない。
\item 原稿はフォントを埋め込んだPDFの提出。
\item 1ファイルは2MB以下。
\end{itemize}

以上のことを守って、ベストを尽くした講演概要の提出を待っている。


%%%%%%%%%%%%%%%%%%%%%%%%%%%% Figure 1 %%%%%%%%%%%%%%%%%%%%%%%%%%%%%%%%%%%%%
\begin{figure}[h]
\begin{center}
\includegraphics[width=5cm]{JPS.eps}
\end{center}
\caption{日本物理学会のマーク。カラー図面が掲載できるようになった。 }
\end{figure}
%%%%%%%%%%%%%%%%%%%%%%%%%%%%%%%%%%%%%%%%%%%%%%%%%%%%%%%%%%%%%%%%%%%%%%%%%%%


\begin{description}
\item[注1:] 現時点では動画、URLへのハイパーリンクは不可である。
\item[注2:] 概要集に掲載された原稿の著作権は日本物理学会に帰属する。
\item[注3:] WEB公開は1年間のみ。オープンアクセスの公開は行わないので、過去の概要集を残したいのであれば記録用DVDの購入を勧める。
\item[注4:] 概要集原稿の提出には登録番号とパスワードを忘れず、原稿〆切に余裕をもってpdf原稿をWebから送信すること。
\end{description}

\end{document}
