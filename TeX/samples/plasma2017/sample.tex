\documentclass[a4paper,twocolumn,fleqn]{article}

\usepackage{txfonts}               % free Times-based fonts for text and math
\usepackage{graphicx}     % dvipdfm or dvipdfmx
\usepackage[absolute]{textpos}                   % PFR style
\usepackage{plasma2017}                   % PFR style

\begin{document}

\title{Instructions for Preparing Manuscripts for the Proceedings of PLASMA2017}
\jtitle{プロシーディングス用カメラレディ-原稿の作り方}

% Use \sup{1},... for superscripts in the authors and affiliations lines.

\author{Toshiro Kaneko\sup{1}, Hiroshi Akatsuka\sup{2}, Mineo Hiramatsu\sup{3}}
\jauthor{金子俊郎\sup{1}, 赤塚洋\sup{2}, 平松美根男\sup{3}}

\affiliation{
  \sup{1}Graduate School of Engineering, Tohoku University,
  \sup{2}Interdisciplinary Graduate School of Science and Engineering,
  Tokyo Institute of Technology,
  \sup{3}Graduate School of Science and Technology, Meijo University}

\jaffiliation{
  \sup{1}東北大学大学院工学研究科,
  \sup{2}東京工業大学大学院総合理工学研究科,
  \sup{3}名城大学大学院理工学研究科}

\begin{abstract}
  The PLASMA2017 proceedings will be printed photographically,
  and thus all manuscripts for the proceedings need to be camera-ready,
  i.e., in a state in which they can be reproduced without any changes.
  The manuscripts submitted will be reproduced directly by photo-offset
  lithography, without any reduction or magnification.
  %
  It is essential that much greater care than usual must be
  taken with the preparation of manuscripts, so that there are no errors,
  smudges, stray marks, or misspellings.
  %
  These instructions have been prepared using the style you should adopt,
  and should be studied closely.
\end{abstract}

\maketitle  % Don't forget to put this!

\begin{textblock}{99}(0.5, 1)
  \textbf{\LARGE{21P-999}}
\end{textblock}

%%%%% MAIN TEXT %%%%%

\section{General}

The authors are requested to submit
\textbf{an electronic file (PDF format) containing a two-page manuscript}
for the PLASMA2017 Proceedings via the symposium web site;
http://www.jspf.or.jp/PLASMA2017/.
%
Please note that only the electronic version of the manuscript
in PDF format will be accepted.
%
The due date for the electronic submission of the manuscript is
\textbf{Oct 31 (Tue.), 2017. }
%
The manuscript should be prepared in English and in a
\textbf{two-column, camera-ready form,}
according to the instructions that are given in detail below.
%
Color figures are acceptable since the proceedings will be distributed
only in electronic files.
The proceedings will be available on your arrival at the conference.

The articles intended to be published in Rapid Communications section
should be short without attempting the completeness as required for a
regular paper.  The length of the manuscript must not exceed 2
pages in this format.  The paper size must be A4 ($21.0\,\mathrm{cm}
\times 29.7\,\mathrm{cm}$).

\section{Typing}
The manuscript should be typed or printed in English
on one side of good quality A4-size (210×297 mm) white paper,
using single spacing with the preferred type font
being 11 point Times New Roman.
%
Please leave 25 mm margins on the top and bottom,
and 20 mm on both the left and right,
so that the manuscript is within a 170×247 mm rectangular area.
%
Standard 8.5×11 inch paper may be used with all margins set to conform to
this 170×247 mm area.
%
The length of a manuscript is strictly limited to 2 pages,
including figures and tables.
%
Manuscripts should be normally prepared with a computer word processor.
%
When that is not possible, carefully typed manuscripts will be accepted.
%
The manuscripts should be justified on both the left and right if possible.
The preferred font is Times New Roman.
If that is not available, choose another proportional space font
as close as possible in appearance.

\section{Title}

The title of the paper should be typed with the initial letter of each word being capitalized, except articles, prepositions, and conjunctions. The title should be centered, and the preferred font is bold-faced Times New Roman 14 point in size. Please write in the title also in Japanese if any.

\section{Author(s) and Affiliation(s)}
Skip one line after the title line(s) and type authors’ names
(speaker underlined), which should preferably be spelled out.
Capitalize the initial letter of the family name and
that of the first name of each author.
%
Then, skip another line before beginning the authors' affiliation(s).
The authors' names and affiliations should also be centered.
%
The preferred font is Times New Roman 12 point in size for authors' names,
and italic Times New Roman 10 point in size for affiliations.
Please write in the author's names and affiliations also in Japanese if any.

\section{Abstract}
Each manuscript should include a short abstract
(normally not more than 100 words).
%
Skip one line after the authors' affiliation line(s),
and type the body of the abstract.
%
Please leave an additional 10 mm margin on both the left and right side.
The preferred type font is 10 point Times New Roman.

\section{Main Body of the Text}
The main body of the text should be divided into sections,
and begin after skipping two lines after the last line of abstract.
The sections should be numbered sequentially.
%
Capitalize the initial letter of each word in the headings, except articles,
prepositions, and conjunctions.
%
The preferred font for the main body of the text is Times New Roman,
11 point in size, with the heading being bold-faced.
%
Indent the first line of each paragraph.
Do not skip a line between paragraphs, but leave a line between sections.
%
Additionally, the text in each section may be divided into subsections,
where the font for the subheadings should be italic
with only the initial letter being capitalized for each subheading.
%

\begin{table}
  \caption{
    \label{tb:sample_table} A sample table
  }
  \begin{tabular}{c c c}
    \hline
    & Intensity (arb. units) & $T_\mathrm{e}$ (eV) \\
    \hline
    Chamber A &       25000 &          4.2\\
    Chamber B &       69800 &          2.5\\
    \hline
  \end{tabular}
\end{table}

Additionally, the text in each section may be divided into subsections,
where the font for the subheadings should be italic
with only the initial letter being capitalized for each subheading.

\subsection{Tables}
Tables should be numbered serially throughout the paper with Roman numerals,
and each should be placed in the text where reference is made to it.
When tables are referred to in the text, they should be typed in full thus:
Table I (i.e., with a single space between Table and the number following).
 Moreover, table headings should always appear above the table.
 The preferred font is 10 point Times New Roman for tables and table headings.

\subsection{Mathematical and structural formulae }
Mathematical and structural formulae should be written with particular care,
and may be numbered:

            (1)

However, simple expressions should be left in the text, written on one line,
e.g., $R=a/(b+c)$.

\subsection{Figures}
Figure should appear as part of the text, inserted where mentioned.
Color figures are acceptable.
They should be numbered serially throughout the paper with Arabic numerals.
%
When figures are referred to in the text, they should be typed thus:
Figure 1 at the beginning of the sentence, while Fig. 1 in the sentence
(with a single space between Figure or Fig. and the number following).
%
Figure captions should be centered below the figure,
with the caption in small letters,
and an initial capital for the first word and proper nouns only.
%
The preferred font is 10 point Times New Roman for figure captions.
In addition, photographs, if used, should be preferably sharp, well-contrasted, glossy prints.

\section*{Acknowledgments}
Acknowledgments, if any,
should be placed at the end of the text before the references.

\section*{References}
References mentioned in the text should be numbered sequentially,
and the number should appear in brackets such as [1], [2,3], and [1-4].
When journals are listed in references, the journal title, volume number,
and year are followed by the inclusive page number.
%
Abbreviations of journal titles should accord with the usage of the Japanese
Journal of Applied Physics (JJAP).
References should be listed at the end of the main body of the text
in numerical order, with the preferred font of 10 point Times New Roman:

\begin{thebibliography}{9}
\bibitem{Kru}
  M. D. Kruskal and R. M. Kulsrud, Phys.\ Fluids
  \textbf{1}, 265 (1958).
\bibitem{Spi}
  L. Spitzer, Jr., \textit{Physics of Fully Ionized Gases}
  (Interscience Publishers, New York, 1959) p.~20.
\bibitem{Tan}
  S. Tanaka, to be published in Jpn.\ J.\ Appl.\ Phys.
  \textbf{22} (1983).
\end{thebibliography}

\end{document}
